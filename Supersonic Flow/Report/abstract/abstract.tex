\documentclass{article}

%\usepackage{revtex4-1}
\usepackage{soul}
\usepackage{color}
\usepackage{outlines}
\usepackage{multirow}
\usepackage{array}
\usepackage{supertabular}
\usepackage{amsmath}
\usepackage[margin=1in]{geometry}
\usepackage{fancyhdr}
\usepackage{setspace}
\let\proof\relax
\let\endproof\relax
\usepackage{amsthm}
\usepackage{amssymb}
\usepackage{indentfirst}
%\usepackage{caption}
\pagestyle{headings}
\usepackage{setspace}
\usepackage[utf8]{inputenc}
%\usepackage[english]{babel}
\usepackage{graphicx}
\usepackage{textcomp}
\usepackage{gensymb}
\usepackage{url}
\usepackage{siunitx}
%\usepackage{subcaption}
\usepackage{float}
%\usepackage{circuitikz}
\usepackage{parskip}
\usepackage{wrapfig, lipsum, booktabs}
\usepackage{boldline}
\usepackage{pdfpages}
\usepackage{rotating}
\usepackage{tikz}
\usepackage[toc,page]{appendix}
\usepackage[colorlinks=false, linkbordercolor = {white}]{hyperref}
\usepackage{cases}
\usepackage{nomencl}
\usepackage{subcaption}
\usepackage{enumitem}
\usepackage{placeins}
\makenomenclature
\makeindex
%\newcommand{\cmmnt}[1]{\ignorespaces}
%\setlength{\belowcaptionskip}{-0.1in}

\title{Numerical Solutions of Transonic Flow Using Mass Conservation and Bernoulli Potential Flow with Artificial Diffusion}

%%% first author
\author{M. Hafez\\\textit{University of California, Davis}\\Andrew M. Chuen\\\textit{University of California, Davis}
%	\affiliation{
%		Student, 999099339\\
%		MAE 210 - Advanced Fluid Mechanics\\
%		Department of Mechanical Engineering\\
%		University of California\\
%		Davis, California 95616\\
%		Email: amchuen@ucdavis.edu
%	}	
}

\begin{document}
\maketitle

\begin{abstract}
	\lipsum[20]
	%Sentence 1:  Big picture topic that is being intensively debated in your field/fields, possibly with reference to scholars (“The question of xxx has been widely debated in xxx field, with scholars such as xxx and xx arguing  xxx]”).
	%With the regained interest in commercial supersonic flight comes a need to efficiently model transonic and supersonic behavior.

	%Sentence 2:  Gap in the literature on this topic.  This GAP IN KNOWLEDGE is very, very bad, and detrimental to the welfare of all right thinking people.  This is the key sentence of the abstract. (“However, these works/articles/arguments/perspectives have not adequately addressed the issue of xxxx.”).
	
	%Sentence 3:  Your project fills this gap (“My paper addresses the issue of xx with special attention to xxx”).
	
	%Sentence 4+ (length here depends on your total word allowance, and more sentences may be possible):  The specific material that you are examining–your data, your texts, etc. ( “Specifically, in my project, I will be looking at xxx and xxx, in order to show xxxx.  I will discuss xx and xx, and juxtapose them against xx and xx, in order to reveal the previously misunderstood connections between xx and xx.”)
	
%	Sentence 5:  Your main argument and contribution, concisely and clearly stated. (“I argue that…”)
	
%	Sentence 6:  Strong Conclusion!  (“In conclusion, this project, by closely examining xxxxx, sheds new light on the neglected/little recognized/rarely acknowledged issue of xxxxx. “).
\end{abstract}

\end{document}